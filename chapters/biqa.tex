
\subsection{Blind Image Quality Assessment (BIQA)}\label{sec:biqa}
\vspace{-3mm}
BIQA is based on metrics described in the ISO standard 29794-6\cite{iso},
and four 3rd party tools; the NIQE\cite{niqe}, BRISQUE\cite{brisque},
JP2KNR\cite{jp2knr}, and BIQI\cite{biqi}.
Table \ref{tab:indclas} shows the different metrics retrieved from
the images, marked with "BIQA". Not all of metrics gathered are used for the process of training and
classifying a SVM, but may be used later and is currently used for retrieving 
and calculating the other ISO metrics.



\paragraph{Usable Iris Area}
The usable iris area, as standardised in section 6.2.1 of the ISO 29794-6
\cite{iso}, is the non-occluded area of the iris.
The method used in this paper is a simple calculation of calculating the
percentage iris that is missing due to the size of the pupil.
\vspace{-3mm}
\begin{align}
	UIA &= (1-\frac{N_{occluded}}{N_{iris}} ) \cdot 100
\end{align}



\paragraph{Iris-Pupil Contrast}
The contrast between the iris and pupil is stated in section 6.2.3 of ISO
29794-6\cite{iso}. 
The implementation uses data sampling from the iris and pupil 
area, not the entire area.  The iris and pupil values are selected by
calculating the median values, then calculating the contrast using these values.
In addition the Weber ratio formula, it implements the same formula as for the 
iris-sclera contrast. Giving an extra metric.
\vspace{-3mm}
\begin{align}
	weber\_ratio &= \frac{|iris - pupil|}{1 + pupil}	\\
	IPC &= \frac{weber}{0.75 + weber}
\end{align}



\paragraph{Iris-Sclera Contrast}
The ISO standard provides a method to compute the contrast between the iris area
and the sclera. By measuring the lightness of the sclera area and the iris 
separately one can compute the contrast ratio, using the formula from the ISO
standard 29794-6 section 6.2.2 \cite{iso}.  This papers approach is the same as
for iris-pupil contrast by using samples instead of the entire area. BIQA then
uses the median values of the samples.
\vspace{-3mm}
\begin{align}
	ISC = \frac{|sclera - iris|}{sclera + iris} \cdot 100
\end{align}




\paragraph{Grey Scale Utilisation (GSU)}
The GSU is measured as the entropy of grey scale pixels and represents
the number of bits needed to represent the image.  The entropy is calculated as
below. Where, $p_i$ is the number of pixels with a given value, $p$ is the
number of all pixels and $P(p_i)$ is the probability of $p_i$ occurring in the 
pixel set.  BIQA calculates the GSU for both the entire image and a subset of 
the iris, giving two separate metrics.
\vspace{-3mm}
\begin{align}
	H &= -\Sigma p_i \log_2 p_i \\
	P(p_i) &= \frac{ \Sigma{p_i} }{ \Sigma{p} }
\end{align}



\paragraph{General Metrics}
In addition to the more specific ISO standard requirements for image quality,
the ISO standard\cite{iso} sets forth additional metrics to assess the quality
of the iris image captured.  Which some are extracted, but not used for this
paper.


