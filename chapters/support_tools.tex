\section{Support tools}
\label{sec:support}

While developing the main tools, I was forced to develop some support tools in
order to manage the databases of images, generating Matlab files, classification
and automation of the segmentation processing with OSIRIS\cite{osiris}.


\subsection{WebIIC}
\label{sec:webiic}
Web Iris Image Classifier (WebIIC)\cite{webiic} performs the task of performing
manual classification of images in the database.  The WebIIC has three tasks;
load the list of images into the database, classification of images, and manual
positioning of the right pupil, left pupil and left iris. These points must be
aligned align along the centre of the pupil along the Y-axis.

The latter task is because of future ability to redo segmentation with fixed
values for pupil and iris size and position.  OSIRIS v4.1 does not allow for
this at the current time.

WebIIC eases the process of classifying images manually significantly and makes
the classification random.  This because it selects the subset of all images not
yet classified, then performs a random selection of one image from this subset.
Which is then loaded into the web frontend for classification or manual point
selection.

The classification allows an image to be set to either of three values,
"Not Classified", "Good", "Bad" or "Error". The error class is used for very
poorly classified images, but is not used in this project.

\begin{description}
\item [Good] The image receives a classification of "1" in the "CLASS" field.
\item [Bad]  The image receives a classification of "0" in the "CLASS" field.
\item [Error] The image receives a classification of "-2" in the "CLASS" field.
\item [Skip] Nothing is done and the image is skipped, maintain current "CLASS"
	which is by default "-1".
\end{description}



\subsection{DBIris}
\label{sec:dbiris}

DBIris\cite{dbiris} consists of several scripts, with the goal of generating a
file containing all the databases.  Which contains all the images successfully
processed by OSIRIS v4.1\cite{osiris}.  The finished files are to be used by
webIIC for loading images into its database and by BIQA and IQA to load images
for metric extraction.

The first script to use is the "osiris\_gen.py" which takes a single file
containing the full path of each image to process.  This will run each image
through the defined configurations for OSIRIS, until the iris is found and
segmented or no more configuration files are available. The resulting images are stored in
the defined folder according to the OSIRIS configuration files.

The second script to run is the "dbGenerator.py" which takes the same file as
"osiris\_gen.py" uses, an integer stating how many databases to create, the path
to the Matlab file containing the databases and the path to the file to be used
by webIIC.

The third script, which is reserved for the future, is the "osiris\_man\_gen.py"
which generates the segmented iris and pupil image(without occlusion assessment)
and the mask for iris area.

DBIris also stores all the images in two folders, the "db\_periocular" and
"img\_processed" by default.  Making it easier to mange for all applications by
having a centralised repository.




