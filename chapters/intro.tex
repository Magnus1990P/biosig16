\section{Introduction}
Current quality metrics(QM) used in iris recognition is based on the Near
Infra-Red (NIR) spectrum.  This paper explored several methods to extract QMs
from an image in the visible wavelength(VW) spectrum. The extracted metrics is 
then used as input to an SVM classifier before the metrics ability to classify
the data set is analysed.

This project was done due to the increasing popularity of using iris recognition
on photos in the VW spectrum.  Therefore it is important to verify the quality
of the image, which this paper has explored.  However, the classification has
not been run through a comparison test because it is out of the scope.

The paper created two tools for extracting QMs from periocular images. "Blind 
Image Quality Assessment" (BIQA) \cite{biqa} extracts twenty-three metrics, of
which a subset are used for assessment.  "Iris Quality Assessment" (IQA)
\cite{iqa} is a reproduction of Hugo Proenca paper "Quality Assessment 
of Degraded iris images acquired in the visible wavelength" \cite{hugo}. IQA 
collects the same metrics as Proenca, except for the iris pigmentation metric.

