
\section{Testing Methodology}\label{sec:test}
\vspace{-3mm}
To perform classification 625 "good" and 560 "bad" images was manually
classified using webIIC \cite{webiic} and used for training the support vector
machine (SVM). The testing was done by using an SVM to classify the individual
metrics on the data sets to ascertain the distribution then analyse the results.

After randomly exploring these data sets, I've noticed a large portion of the
images are of lower quality. After classification the result was plotted using a
scatter-plot diagram for illustrative purposes which shows the overlap and can
be used for further analysis.




\section{Data sets}\label{sec:dataset}
\vspace{-3mm}
Three databases was used for this paper and consisted of totally 11.292 images
of the periocular region of the face.  Originally it was 12.689, but some images
could not be used because of errors occurring in OSIRIS v4.1\cite{osiris}.
The data sets which has been used in this paper is the UBIRIS.v2\cite{ubiris},
MICHE\cite{miche} and an independent data set captured at Norwegian University of
Technology and Science (NTNU).

UBIRISv2. consists of images captured at a distance during movement. It was 
created by Hugo Proenca, et al \cite{ubiris}. The images are 600x400 pixels and
of varying quality. Most images are focused on the periocular region, but
several have failed to capture the iris.  The data set I've used contained a
total of 11101 images.
The MICHE\cite{miche} data set contains large images of the periocular region. 
The images are captured in various environments using either an iPhone 5 or 
Samsung Galaxy using both cameras on each phone and in two environments for each
camera. MICHE contains images of varying quality and resolution. In total I've
used 1536 of the images from MICHE.
The last data set is generated at NTNU and contains 52 high quality images of
uniform lighting.

