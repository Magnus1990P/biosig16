\section{Related work}
\vspace{-4mm}
This paper has explored and developed two systems for extracting blind quality 
features from periocular images captured in the visible wavelength (VW)
spectrum.
The resulting features is then run through a SVM for training and classification
where the training set was manually classified.

First off, BIQA\cite{biqa} is the system which was based upon metrics gathered
from the ISO standard\cite{iso} and various methods as detailed below. Which
include the tools NIQE \cite{niqe}, BRISQUE \cite{brisque}, NSS \cite{nss},
JP2KNR \cite{jp2knr} and BIQI\cite{biqi}.

It has already been done work on blind image quality assessment, especially the 
University of Texas\footnote{University of Texas:
\texttt{http://live.ece.utexas.edu/research/Quality/index.htm}}.
The paper has implemented the following blind image quality assessment (BIQA)
tools that has been published and are publicly available; BRISQUE, JP2KNR
Quality metric, BIQI, and NIQE.

Anish Mittal, Rajiv Soundararajan, and Alan C. Bovik\cite{niqe} has developed a
blind image quality model called "Natural Image Quality Evaluator" (NIQE) which
runs through an image and assess its quality based on a statistical models for
the natural scene statistics (NSS) based on L. Ruderman article "The Statistics
of Natural Images"\cite{nss}.

Anish Mittal, Anush K. Moorthy, and Alan C. Bovik has proposed a BIQA model
operating on the spacial domain, "Blind/Referenceless Image Spatial Query
Evaluator" (BRISQUE)\cite{brisque}.  This model focuses on using scene
statistics to quantify possible losses of "naturalness", resulting in a holistic
measure of quality.

Hamid R. Sheikh, Alan C. Bovik and Lawrence Cormack \cite{jp2knr} has proposed a
method of applying NSS to measure the quality of images compressed using wavelet
based image compression.  The hypothesis they started with was that the
compression results in loss of quality that can be related to human perception
of quality.

The tool "Blind Image Quality Index" (BIQI)\cite{biqi} developed by Anush
K. Moorthy and Alan C. Bovik is focusing on assessing the distortions
in an image. BIQI is based on a trained classifier that does not require any
prior knowledge once its trained and can be extended to assess any type of
distortion.  As with the NIQE\cite{niqe}, BIQI is based on an NSS model.

For the other metrics being implemented in this project, I have relied on the
ISO 29794-6 standard\cite{iso}.

For the second tool, the Iris Quality Assessment (IQA)\cite{iqa}, I've reversed
engineered Hugo Proencas paper\cite{hugo}.  Many of the points in this
paper are overlapping with the metrics from the ISO standard.  Hugo proposes to
use seven metrics as the basis for classification. This paper implemented
six of them.


