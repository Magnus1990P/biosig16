\subsection{Result}
\label{sec:res}
After inspecting the individual metrics ability to perform classification based
on the 625/560 images used for training is quite well.  During my initial
analysis of the metric distributions before classification, I believed several
of the would do well.  The resulting scatter plots from the classifications
shows that the 3rd party tools work very well on classifying the images as good
or bad with minimal training.

The 1185 images used for training represent a portion of ~10.5\%\ of all images
used. This 5.5\%\ "good" images was used to train against "good" image metrics,
while 4.9\%\ "bad" images was used to train against "bad" image metrics.

This is quite small number and should be increased in future tests to verify the
results.

The metrics which performed good on classification was, BIQA Iris-Pupil contrast
metric (fig. \ref{fig:clas_ipc}), BIQA Iris-Sclera contrast metric (fig.
\ref{fig:clas_isc}) BIQA Pupil-iris ratio metric(fig. \ref{fig:clas_pir}), BIQA
GSU metric(fig. \ref{fig:clas_gsu}), NIQE\cite{niqe} (fig. \ref{fig:clas_niqe}),
BRISQUE\cite{brisque} (fig. \ref{fig:clas_brisque}), JP2KNR\cite{jp2knr}(fig.
\ref{fig:clas_jp2knr}), and BIQI\cite{biqi} (fig. \ref{fig:clas_biqi}).

The result is that this subset of metrics of "good" metrics are very likely to
be applicable for images captured in the visible wavelength spectrum.  The
remaining metrics are due to poor implementation, or that the data not usable
for classifying images of irises in the visible wavelength spectrum.

There was many hinders in order to perform this project, the level of Hugo
Proencas paper\cite{hugo} which was too advanced at some points in the project,
OSIRISv4.1\cite{osiris}, which has many bugs at its current state.  Segmentation
fault, unable to discovering pupil and iris boundaries on high resolution and
high quality images.  This begs the question whether OSIRISv4.1 in itself is
applicable for performing pupil and iris discovery as well as segmentation on
images in visible wavelength spectrum.  The last was the state of the data sets,
many images don't have irises visible in the image, way to occluded or otherwise
degraded.



